\documentclass[12pt]{article}
\usepackage[utf8]{inputenc}
\usepackage{ctex}
\usepackage{amsfonts}
\usepackage{setspace}
\usepackage{fontspec}
\setmainfont{Times New Roman}
\usepackage{xeCJK}
\usepackage{graphicx} % Required for inserting images
\onehalfspacing
\usepackage{geometry}
\geometry{a4paper, margin=0.8in}
\title{调研提纲}
\author{郑姜崇杰、高琛、王楚怡}
\date{2024年7月}

\begin{document}

\maketitle

\section{基本信息/ Demographic Info.}
\begin{enumerate}
    \item 访问年份/ cyear: 2024
    \item 访问月份/ cmonth: 07
    \item 家户编号/ fid: (如第i个家户编号为700i,表示为第七组采访的第i个家户)
    \item 个人编号/ pid: (如第i个家户第j个个体编号为700i0j)
    \item 性别/ Gender: 男/女
    \item 出生年份/ BirthY: 
    \item 出生月份/ BirthM: 
    \item 年龄/ Age: (自动计算,注意几个年龄断点,如35岁门槛,女性50/55岁,男性60岁等)
    \item 民族/ Ethnicity: 
    \item 出生地/ BirthPlace: (是否是迁移过来的,在当地有没有较深厚的社会关系)
    \item 兄弟姐妹数量/ Siblings: (包含已经去世的) 
    \begin{itemize}
        \item 在当地的兄弟姐妹数量/ LocalSiblings: 
        \item 不在当地的兄弟姐妹数量/ NotLocalSiblings: 
    \end{itemize}
    \item 亲戚数量/ Relatives: 
    \begin{itemize}
        \item 有多少关系亲近的/ How many close ones: 
    \end{itemize}
    \item 教育背景/ Educ: 幼儿园/小学/初中/中专/高中/大专/本科/研究生/博士
    \item 父亲教育背景/ Feduc: 幼儿园/小学/初中/中专/高中/大专/本科/研究生/博士
    \item 母亲教育背景/ Meduc: 幼儿园/小学/初中/中专/高中/大专/本科/研究生/博士
    \item 婚姻状态/ Marriage: 未婚/已婚/离婚
    \item 子女情况/ Kids: 数量
\end{enumerate}

\section{工作情况相关/ Work Related}
\begin{enumerate}
    \item 是否正在工作/ Working: 是/否

\end{enumerate}

\subsection*{就业地点去向}
\begin{enumerate}
    \item 您目前的工作地点是在城市还是农村,具体在哪个城市或哪个农村?
    \item 您是否有计划在将来一段时间内离开当前工作地点,去往其他地方?
    \item 您选择当前工作地点的主要原因是什么(例如:就业机会、家庭原因、生活成本等)?
    \item 您认为工作地点对您的生活质量和职业发展有何影响?
\end{enumerate}

\subsection*{就业途径}
\begin{enumerate}
    \item 您是通过什么方式找到当前工作的(例如:亲友介绍、招聘广告、职业介绍所等)?
    \item 在寻找工作过程中,您遇到了哪些困难或挑战?
    \item 您认为哪些因素对您获得就业机会最为关键?
\end{enumerate}

\subsection*{行业分布}
\begin{enumerate}
    \item 您目前从事的是哪个行业?
    \item 您选择该行业工作的主要原因是什么?您认为该行业对农村居民的吸引力在哪里?
    \item 您从事的行业对教育要求和工作经验要求如何?
    \item 您如何看待当前行业的发展状况和未来趋势,它对农村劳动力的需求有何变化?
\end{enumerate}

\subsection*{年龄特点}
\begin{enumerate}
    \item 您多大年龄?您认为年龄对您就业选择和工作机会有何影响?
    \item 在您的年龄段中,非农就业的常见职业有哪些?您认为这些职业对年龄有何特定要求?
    \item 您觉得当前的就业市场是否对不同年龄段的农村居民提供了足够的就业机会和职业发展路径?如果没有,是哪一部分人没法找到工作?
\end{enumerate}

\subsection*{性别特点}
\begin{enumerate}
    \item 您是否觉得性别对您在工作场所的晋升机会或工作类型有影响?
    \item 在您的经验中,男性和女性在非农就业领域中面临的主要挑战有何不同?
\end{enumerate}

\subsection*{合同条款}
\begin{enumerate}
    \item 您在签订劳动合同时,是否有机会详细了解合同内容,包括工作职责、工作时间和解雇条件等?
    \item 您的劳动合同中是否明确规定了试用期、合同期限以及续约条件?
    \item 如果遇到劳动争议,您的劳动合同是否提供了解决争议的途径或机制?
\end{enumerate}

\subsection*{休假}
\begin{enumerate}
    \item 您的工作是否提供法定的带薪年假?如果是的话,您每年能享受多少天的带薪休假?
    \item 在实际工作中,您是否能够自由安排带薪休假?
    \item 您是否了解带薪休假的相关法律法规?在休假权益受到侵犯时,您知道如何维护自己的权益吗?
\end{enumerate}

\subsection*{工资水平}
\begin{enumerate}
    \item 您目前的月收入或日收入大约是多少?这与您所在地区的平均收入水平相比如何?
    \item 您认为当前的工资水平是否能够满足您和家庭的基本生活需求?
    \item 您是否收到过任何形式的奖金或补贴?这些额外收入对您的生活质量有何影响?
\end{enumerate}

\section{非农就业的社会保障}

\subsection*{社会保险}
\begin{enumerate}
    \item 除了基本的养老保险和医疗保险,您是否享有失业保险、工伤保险和生育保险等其他社会保险?
    \item 您是否了解如何申请和使用这些社会保险?在实际操作中遇到过哪些困难?
    \item 您认为现有的社会保险体系是否充分覆盖了您在非农就业中可能面临的风险?
\end{enumerate}

\subsection*{子女教育}
\begin{enumerate}
    \item 您的子女目前在何处接受教育?
    \item 您或您的配偶是否会因为子女教育而考虑更改就业?
    \item 您是否面临因非农就业而产生的子女教育费用压力?
    \item 您认为非农就业对子女接受的教育质量有何影响?是否有政策或措施帮助您解决这一问题?
\end{enumerate}

\subsection*{个人养老}
\begin{enumerate}
    \item 您目前是否有参与任何形式的养老保险计划?这些计划是否足以满足您对未来养老的期望?
    \item 参与非农就业而非传统农业对您的养老计划有何影响?您是否担心退休后的生活保障?
    \item 您是否了解政府提供的农村养老保险政策?这些政策在实际操作中是否容易理解和使用?
\end{enumerate}

\subsection*{老人赡养}
\begin{enumerate}
    \item 您是否需要赡养家中的老人?这一责任是否影响了您选择就业地点或就业类型的决定?
    \item 在考虑就业机会时,您是否会因为需要照顾家中老人而拒绝一些长时间连续的工作?
    \item 如果将老人接入工作地,老人是否能享有更优的医疗服务与医疗保障?
\end{enumerate}

\subsection*{医疗保险}
\begin{enumerate}
    \item 您在非农就业中是否享有医疗保险?这些保险是否覆盖了您和家庭成员的主要医疗需求?
    \item 您在遇到重大疾病时,医疗保险是否能够提供足够的帮助?您是否有其他方式来应对医疗费用?
    \item 您是否了解农村合作医疗或其他医疗补助政策?这些政策在实际中对您的医疗保障有何帮助?
\end{enumerate}

\section{非农就业的发展形势}

\subsection*{主要就业行业的发展趋势}
\begin{enumerate}
    \item 您认为近年来哪些非农行业提供了更多的就业机会?这些行业为何吸引您或您周围的人?
    \item 您观察到在非农就业中,有哪些行业的收入和工作条件有了显著改善?
    \item 您如何看待当前政策对农业转移人口就业的支持,以及这些政策如何影响您的就业选择?
\end{enumerate}

\subsection*{传统非农就业与新型非农就业}
\begin{enumerate}
    \item 您或您认识的外出务工人员在传统非农行业(如制造业、建筑业)与新型非农行业(如服务业、电子商务)之间是如何做出选择的?
    \item 您是否注意到,随着技术进步和市场变化,非农就业出现了哪些新形式或新机会?
    \item 您认为新型非农就业与传统非农就业相比,在稳定性、收入水平和工作条件上有哪些优势和挑战?
\end{enumerate}

\subsection*{旅游业下的非农就业}
\begin{enumerate}
    \item 您是否发现旅游业的发展为当地居民提供了新的就业机会?这些机会主要体现在哪些方面?
    \item 随着乡村旅游的兴起,您认为这对提高农村居民的非农就业有哪些积极影响?
    \item 您如何看待政府在推动旅游业发展中的作用,以及这些措施如何帮助改善您的就业状况?
\end{enumerate}

\subsection*{城市回流人员的成因与就业}
\begin{enumerate}
    \item 您能分享一下是什么原因促使您从城市回到农村的吗?(例如:家庭原因、工作环境、生活成本等)
    \item 回流后,您在本地找到的工作类型与您在城市时的工作有何不同?您是如何适应这些变化的?
    \item 您认为本地的就业机会是否能满足您的需求和期望?在就业方面,您觉得政府或社区还能提供哪些帮助或服务?
\end{enumerate}

\end{document}
