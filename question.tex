\documentclass[12pt]{article}
\usepackage[utf8]{inputenc}
\usepackage{ctex}
\usepackage{amsfonts}
\usepackage{setspace}
\usepackage{fontspec}
\setmainfont{Times New Roman}
\usepackage{xeCJK}
\usepackage{graphicx} % Required for inserting images
\onehalfspacing
\usepackage{geometry}
\geometry{a4paper, margin=0.8in}
\usepackage[style=chinese-erj, backend=biber]{biblatex}
\addbibresource{ref.bib}  


\title{非农就业}
\author{郑姜崇杰、高琛、王楚怡}
\date{2024年7月}

\begin{document}

\maketitle

\section{研究概述}
\subsection*{研究目的及意义}
在全球化和城市化进程中,农村地区劳动力向非农产业的转移是经济发展的必然结果。这一现象不仅关系到农村居民的经济福祉和社会地位的提升,而且对于促进农业现代化、优化劳动力结构、推动区域经济均衡发展具有重要作用。

此外,农村非农就业的深入研究有助于揭示劳动力市场动态、就业政策效果以及农村社会经济结构的变迁。通过这一研究,可以为政策制定者提供实证基础,以制定更加精准有效的就业促进政策,促进农村劳动力的稳定就业和高质量就业。同时,这也为理解农村社会经济转型提供了重要的视角,有助于构建包容性增长模式,实现经济、社会和环境的可持续发展。因此,对农村非农就业问题的研究不仅是学术探索的需要,更是社会发展和政策制定的迫切需求。

我们本次研究旨在通过全面深入的访谈,获得详尽的一手非农就业信息,了解实际生活中的非农就业发展状态与问题。通过我们的分析为非农就业政策制定与实际实施提供有价值的真实证据。

\subsection*{过往文献}
非农就业作为农村劳动力转型和城乡经济一体化的关键途径,一直是学术界关注的焦点。\cite{zhang2018off}发现从1978年改革开放开始,中国农村劳动力参与非农就业的比例显著上升,这一趋势一直持续到21世纪10年代中期。近年来,随着城市化的推进和经济结构的调整,农村劳动力向非农产业转移的现象日益显著\cite{sheng2022boosting}。非农就业不仅能有效提高农村居民的收入水平,缩小城乡差距,还对促进农村社会经济发展和实现区域均衡具有重要作用。

从宏观角度来看,\cite{钟甫宁2009土地产权}发现如果没有非农就业机会,土地产权、土地买卖和租赁本身并不会扩大农户的土地经营规模,也不会刺激农业投资。\cite{张平1998中国农村居民区域间收入不平等与非农就业}发现农村非农就业发展,特别是乡镇工业的发展是导致区域间收入不平等的主要因素。

在微观层面,\cite{赵耀辉1997中国农村劳动力流动及教育在其中的作用}发现年轻、未婚、受过较好教育的男性更倾向于外出就业。此外,土地拥有量越少的家庭中的劳动力外出概率越大,而教育程度对外出的影响较小。\cite{王卫东2020人力资本}表明政治资本,如家庭中是否有人担任村干部,虽然对非农就业有积极影响,但这种影响在逐年下降,显示出农村劳动力市场日趋完善。同时\cite{zhang2018off}发现非农就业的类型发生了变化,特别是工资性就业(包括迁移和非迁移)的比例显著增加,而自我雇佣的比例有所下降。

同时,非农就业也对传统农业部门有所影响,\cite{wang2017off}发现,随着非农就业机会的增加,留在农业领域的农民开始减少种植作物的种类,专注于主要作物的种植,从而提高了农业生产的专业化水平。

综上所述,非农就业是推动农村经济社会发展的重要力量。我们的研究需要进一步探讨非农就业与农村社会经济结构变迁的关系,以及如何通过政策引导和市场机制优化非农就业的质量和结构,以实现农村经济的可持续发展。

\section{研究内容}
本研究围绕农村非农就业人群展开,探究非农就业人群的职业选择和未来职业发展,以及非农就业对他们生活各方面产生的影响。
\subsection{工作情况}
非农就业的就业选择具有多样性和复杂性,其所处的不同行业对于劳动力个体的要求差异性也很大
\section{采}
\section{基本信息 / Demographic Info.}
\begin{enumerate}
    \item 访问年份 / cyear: 2024

    \item 访问月份 / cmonth: 07

    \item 家户编号 / fid: 

    \item 个人编号 / pid: 

    \item 性别 / Gender: \\
    \textit{您的性别是什么?}
   \item 出生年份 / BirthY: \\
    \textit{您是哪一年出生的?}
    \item 出生月份 / BirthM: \\
    \textit{您出生的月份是几月?}
    \item 年龄 / Age: \\
    \textit{您现在多大年龄?}
    \item 民族 / Ethnicity: \\
    \textit{您属于哪个民族?}
    \item 出生地 / BirthPlace: \\
    \textit{您的出生地在哪里?您是否从其他地方迁移过来?}
    \item 兄弟姐妹数量 / Siblings: \\
    \textit{您总共有多少兄弟姐妹?(包括已经去世的)}
    
    \begin{enumerate}
        \item 在当地的兄弟姐妹数量 / LocalSiblings: \\
        \textit{您有多少兄弟姐妹目前和您居住在同一地区?}
        \item 不在当地的兄弟姐妹数量 / NotLocalSiblings: \\
        \textit{您有多少兄弟姐妹不住在您所在的地区?}
    \end{enumerate}
    \item 亲戚数量 / Relatives: \\
    \textit{您有多少亲戚?}
    
    \begin{enumerate}
        \item 有多少关系亲近的 / How many close ones: \\
        \textit{您有多少关系亲近的亲戚?}
    \end{enumerate}
    
    \item 教育背景 / Educ: \\
    \textit{您接受的最高教育程度是什么?}
    \item 父亲教育背景 / Feduc: \\
    \textit{您父亲接受的最高教育程度是什么?}
    \item 母亲教育背景 / Meduc: \\
    \textit{您母亲接受的最高教育程度是什么?}
    \item 婚姻状态 / Marriage: \\
    \textit{您目前的婚姻状况是什么?}
    \item 子女情况 / Kids: \\
    \textit{您有多少子女?}
\end{enumerate}
\section{工作情况相关/ Work Related}
\begin{enumerate}
    \item 是否正在工作/ Working: 是/否

\end{enumerate}

\subsection*{就业地点去向}
\begin{enumerate}
    \item 您目前的工作地点是在城市还是农村,具体在哪个城市或哪个农村?
    \item 您是否有计划在将来一段时间内离开当前工作地点,去往其他地方?
    \item 您选择当前工作地点的主要原因是什么(例如:就业机会、家庭原因、生活成本等)?
    \item 您认为工作地点对您的生活质量和职业发展有何影响?
\end{enumerate}

\subsection*{就业途径}
\begin{enumerate}
    \item 您是通过什么方式找到当前工作的(例如:亲友介绍、招聘广告、职业介绍所等)?
    \item 在寻找工作过程中,您遇到了哪些困难或挑战?
    \item 您认为哪些因素对您获得就业机会最为关键?
\end{enumerate}

\subsection*{行业分布}
\begin{enumerate}
    \item 您目前从事的是哪个行业?
    \item 您选择该行业工作的主要原因是什么?您认为该行业对农村居民的吸引力在哪里?
    \item 您从事的行业对教育要求和工作经验要求如何?
    \item 您如何看待当前行业的发展状况和未来趋势,它对农村劳动力的需求有何变化?
\end{enumerate}

\subsection*{年龄特点}
\begin{enumerate}
    \item 您认为年龄对您就业选择和工作机会有何影响?
    \item 在您的年龄段中,非农就业的常见职业有哪些?您认为这些职业对年龄有何特定要求?
    \item 您觉得当前的就业市场是否对不同年龄段的农村居民提供了足够的就业机会和职业发展路径?如果没有,是哪一部分人没法找到工作?
\end{enumerate}

\subsection*{性别特点}
\begin{enumerate}
    \item 您是否觉得性别对您在工作场所的晋升机会或工作类型有影响?
    \item 在您的经验中,男性和女性在非农就业领域中面临的主要挑战有何不同?
\end{enumerate}

\subsection*{合同条款}
\begin{enumerate}
    \item 您在签订劳动合同时,是否有机会详细了解合同内容,包括工作职责、工作时间和解雇条件等?
    \item 您的劳动合同中是否明确规定了试用期、合同期限以及续约条件?
    \item 如果遇到劳动争议,您的劳动合同是否提供了解决争议的途径或机制?
\end{enumerate}

\subsection*{休假}
\begin{enumerate}
    \item 您的工作是否提供法定的带薪年假?如果是的话,您每年能享受多少天的带薪休假?
    \item 在实际工作中,您是否能够自由安排带薪休假?
    \item 您是否了解带薪休假的相关法律法规?在休假权益受到侵犯时,您知道如何维护自己的权益吗?
\end{enumerate}

\subsection*{工资水平}
\begin{enumerate}
    \item 您目前的月收入或日收入大约是多少?这与您所在地区的平均收入水平相比如何?
    \item 您认为当前的工资水平是否能够满足您和家庭的基本生活需求?
    \item 您是否收到过任何形式的奖金或补贴?这些额外收入对您的生活质量有何影响?
\end{enumerate}

\section{非农就业的社会保障}



\subsection*{子女教育}
\begin{enumerate}
    \item 您的子女目前在何处接受教育?
    \item 如果子女不在身边:您平时如何关心子女成长,监督子女教育?您或您的配偶以后是否考虑过将子女接到身边?是否会因为子女教育而考虑更改就业?
    \item 如果子女在身边:您是否面临因非农就业而产生的子女教育费用压力?
    \item 您认为非农就业对子女接受的教育质量有何影响?是否有政策或措施帮助您解决这一问题?
\end{enumerate}

\subsection*{个人养老}
\begin{enumerate}
    \item 您目前是否有参与任何形式的养老保险计划?这些计划是否足以满足您对未来养老的期望?
    \item 参与非农就业而非传统农业对您的养老计划有何影响?您是否担心退休后的生活保障?
    \item 您是否了解政府提供的农村养老保险政策?这些政策在实际操作中是否容易理解和使用?
\end{enumerate}

\subsection*{老人赡养}
\begin{enumerate}
    \item 您是否需要赡养家中的老人?这一责任是否影响了您选择就业地点或就业类型的决定?
    \item 在考虑就业机会时,您是否会因为需要照顾家中老人而拒绝一些长时间连续的工作?
    \item 如果将老人接入工作地,老人是否能享有更优的医疗服务与医疗保障?
\end{enumerate}

\subsection*{医疗保险}
\begin{enumerate}
    \item 您在非农就业中是否享有医疗保险?这些保险是否覆盖了您和家庭成员的主要医疗需求?
    \item 您在遇到重大疾病时,医疗保险是否能够提供足够的帮助?您是否有其他方式来应对医疗费用?
    \item 您是否了解农村合作医疗或其他医疗补助政策?这些政策在实际中对您的医疗保障有何帮助?
\end{enumerate}

\subsection*{社会保险}
\begin{enumerate}
    \item 除了基本的养老保险和医疗保险,您是否享有失业保险、工伤保险和生育保险等其他社会保险?
    \item 您是否了解如何申请和使用这些社会保险?在实际操作中遇到过哪些困难?
    \item 您认为现有的社会保险体系是否充分覆盖了您在非农就业中可能面临的风险?
\end{enumerate}


\section{非农就业的发展形势}

\subsection*{发展形势下个体改变}
\begin{enumerate}
    \item 您就业以来发生了职业发生过变化吗?为什么会考虑从之前的职业转换到现在的?
    \item 您观察过周边的非农就业人群的职业变化吗?大家为什么去这些新的行业?
    \item 在行业选择改变的过程中,您遇到的最大的挑战是什么,新行业对您有新的要求吗?您是如何克服挑战的?
\end{enumerate}

\subsection*{主要就业行业的发展趋势}
\begin{enumerate}
    \item 您认为近年来哪些非农行业提供了更多的就业机会?这些行业为何吸引您或您周围的人?
    \item 您观察到在非农就业中,有哪些行业的收入和工作条件有了显著改善?
    \item 您如何看待当前政策对农业转移人口就业的支持,以及这些政策如何影响您的就业选择?
\end{enumerate}

\subsection*{传统非农就业与新型非农就业}
\begin{enumerate}
    \item 您或您认识的外出务工人员在传统非农行业(如制造业、建筑业)与新型非农行业(如服务业、电子商务)之间是如何做出选择的?
    \item 您是否注意到,随着技术进步和市场变化,非农就业出现了哪些新形式或新机会?
    \item 您认为新型非农就业与传统非农就业相比,在稳定性、收入水平和工作条件上有哪些优势和挑战?
\end{enumerate}

\subsection*{旅游业下的非农就业}
\begin{enumerate}
    \item 您是否发现旅游业的发展为当地居民提供了新的就业机会?这些机会主要体现在哪些方面?
    \item 随着乡村旅游的兴起,您认为这对提高农村居民的非农就业有哪些积极影响?
    \item 您如何看待政府在推动旅游业发展中的作用,以及这些措施如何帮助改善您的就业状况?
\end{enumerate}

\subsection*{城市回流人员的成因与就业}
\begin{enumerate}
    \item 您能分享一下是什么原因促使您从城市回到农村的吗?(例如:家庭原因、工作环境、生活成本等)
    \item 回流后,您在本地找到的工作类型与您在城市时的工作有何不同?您是如何适应这些变化的?
    \item 您认为本地的就业机会是否能满足您的需求和期望?在就业方面,您觉得政府或社区还能提供哪些帮助或服务?
\end{enumerate}

\printbibliography
\end{document}
